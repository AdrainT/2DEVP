\documentclass{article}
\usepackage{epsfig}
\usepackage{amsmath}
\usepackage{amsfonts}
\usepackage{amsthm}
\usepackage{algorithm}
\usepackage{algorithmic}
\usepackage{graphicx}
\usepackage{subfigure}
\newcommand{\vect}{{\bf vec}}
\newcommand {\C}        {{\mathbb{C}}}
\newcommand {\R}        {{\mathbb{R}}}
\newcommand {\N}        {{\mathbb{N}}}
\newcommand {\Z}        {{\mathbb{Z}}}
\newcommand {\Lb}        {{\mathbb{L}}}

 \newtheorem{theorem}{Theorem}[section]
 \newtheorem{lemma}[theorem]{Lemma}
 \newtheorem{proposition}[theorem]{Proposition}
 \newtheorem{corollary}[theorem]{Corollary}
 \newtheorem{definition}[theorem]{Definition}

 \newenvironment{example}[1][Example]{\begin{trivlist}
 \item[\hskip \labelsep {\bfseries #1}]}{\end{trivlist}}
 \newenvironment{remark}[1][Remark]{\begin{trivlist}
 \item[\hskip \labelsep {\bfseries #1}]}{\end{trivlist}}


\setlength{\textwidth}{5.6in}
\setlength{\textheight}{9in}
\setlength{\oddsidemargin}{0.5in}
\setlength{\evensidemargin}{0.5in}

\bibliographystyle{plain}

\title{ Usage }

\begin{document}
\maketitle

\large

\noindent
\texttt{[f,z] = routinename(funname,d,j,C,pars)}

\vskip 2ex

\noindent
funname (string) : \\
the name of the routine that computes
the matrix-valued function $A(\omega)$ and its derivative
at a given $\omega$.

\vskip 1ex

\noindent
d (integer) : \\
number of parameters, i.e., $\omega \in {\mathbb R}^d$

\vskip 1ex

\noindent
j (integer) : \\
minimize/maximize the $j$ largest/smallest eigenvalue/singular value. 

\vskip 1ex

\noindent
C (cell array) : \\
$C\{ k \} = A_k$ for $k = 1,\dots, \kappa$ where $A_k$ are matrices
in the definition of $A(\omega)$.

\vskip 1ex

\noindent
pars (structure) : \\
parameters, pars.bounds.lb and pars.bounds.ub must contain
the extreme corners of the box over which the optimization will be performed.
For instance pars.bounds.lb = [-5 -5] and pars.bounds.ub = [5 5] means that
perform optimization on $[-5,5]\times [-5,5]$.



\vskip 10ex

\noindent
For instance to compute the numerical radius of $A$
type \\
\texttt{>> pars.bounds.lb = 0} \\
\texttt{>> pars.bounds.ub = 2*pi} \\
\texttt{>> C}$\{1\}$\texttt{ = A} \\
\texttt{>> [f,z] = leigopt}$\_$\texttt{max('numrad',1,1,C,pars)}

\vskip 5ex
\noindent
Here numrad is as follows.
\begin{verbatim}
function [M,Md] = numrad(z,C)

M  = (C{1}*exp(i*z) + C{1}'*exp(-i*z))/2;
Md = (i*C{1}*exp(i*z) - i*C{1}'*exp(-i*z))/2; 

return;
\end{verbatim}




\end{document}